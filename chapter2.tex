% Chapter 2: LLMs as Tool Users: Extending Capabilities
\chapter{LLMs as Tool Users: Extending Capabilities}

While LLMs possess remarkable abilities in understanding and generating text, their knowledge is inherently limited by their training data, 
which has a specific cutoff point in time. They also lack the ability to perform precise calculations reliably or interact directly with the external world, 
such as accessing live databases or triggering actions in other software systems. "Tool use," often implemented via "function calling," is a 
mechanism that allows LLMs to overcome these limitations by integrating with external functions, APIs, and data sources.

\section{The Concept of Tool Use (Function Calling)}

The core idea behind LLM tool use is to enable the model to request the execution of predefined functions when it determines that doing so would help 
fulfill a user's request or achieve a given task. Instead of trying to generate an answer based solely on its internal knowledge, the LLM can recognize when it needs, 
for example, current weather information, the result of a mathematical calculation, or data from a specific enterprise API.

Function Calling is the common term for how this is implemented. In this paradigm, LLMs like GPT-4 have been fine-tuned to detect when a function call is necessary 
and to output a structured data object, typically in JSON format, that specifies the name of the function to be called and the arguments to pass to it. 
Crucially, the LLM itself does not execute the function. Instead, it acts as a "reasoner" or "router," deciding which tool (function) is appropriate for 
the current context and what inputs that tool needs. The actual execution of the function is handled by the application code that is interacting with the LLM.

This capability represents a fundamental architectural pattern for building LLM-powered applications. It allows software engineers to create a powerful synergy 
by combining the natural language understanding and reasoning strengths of LLMs with the deterministic, precise, and specialized capabilities of existing software 
tools and APIs. Each component in this architecture focuses on what it does best: the LLM interprets intent and plans the use of tools, while traditional software 
components execute those tools and provide factual results. This separation of concerns is key to building robust, extensible, and reliable LLM-integrated systems.

\section{How LLMs Interact with External APIs and Functions}

The interaction between an LLM and an external tool or API typically follows a well-defined workflow:

\begin{enumerate}[label=\arabic*.]
    \item \textbf{User Query to LLM:} The process starts with a user providing a query or instruction to the LLM-powered application.
    \item \textbf{LLM Processing and Tool Selection:} The LLM analyzes the query. If it determines that an external tool is needed to fulfill the request, 
    it generates a structured output (e.g., a JSON object) that specifies the function to be called and the necessary arguments. For example, if the user asks, 
    "What's the weather like in London?", the LLM might output:
	\begin{lstlisting}[ caption=Example LLM output for tool selection, basicstyle=\ttfamily\footnotesize\linespread{0.9}, breaklines=true]
		{
			"tool_name": "get_weather",
			"arguments": {
				"city": "London",
				"unit": "celsius"
			}
		}
	\end{lstlisting}
    \item \textbf{Application Code Receives Request:} The application code that hosts the LLM receives this structured request.
    \item \textbf{Function Execution by Application:} The application code then parses this request and executes the actual \texttt{get\_weather} function 
    (which might involve calling an external weather API) with the provided arguments ("London", "celsius").
    \item \textbf{Result Returned to LLM:} The result from the executed function (e.g., "The weather in London is 15°C and cloudy") is then sent back to the 
    LLM, typically as part of a new prompt or context.
    \item \textbf{LLM Formulates Final Response:} The LLM uses this new information (the tool's output) to formulate a natural language response to the user's 
    original query (e.g., "The current weather in London is 15 degrees Celsius and cloudy.").
\end{enumerate}

To make the LLM "aware" of the available tools and how to use them, developers provide descriptions or schemas of these tools as part of the initial prompt or 
system message to the LLM. These descriptions detail what each tool does, the parameters it expects (including their names, types, and purpose), and sometimes 
even examples of how to use it.

The quality of these tool descriptions is critical. They act as an "API contract" for the LLM. A clear, accurate, and unambiguous description directly impacts 
the LLM's ability to correctly choose the right tool and provide the correct arguments. If the description is vague or misleading, the LLM is likely to make 
errors in its function call requests, leading to application failures or incorrect results. Therefore, crafting effective tool descriptions is a new and important 
skill for software engineers building LLM applications, akin to designing good API documentation for human developers.

\section{Practical Examples of LLM Tool Integration}

The ability of LLMs to use tools unlocks a vast range of practical applications, making them significantly more useful in real-world business scenarios. Some 
common examples include:

\subsection*{Retrieving Real-Time Information}
LLMs have knowledge cutoffs, meaning their training data only goes up to a certain point in time. Tools allow them to access current information.
\begin{itemize}
    \item \textbf{Example:} A user asks, "What's the current stock price for Company X?" The LLM can use a \texttt{get\_stock\_price} tool that calls a financial
     markets API to fetch the latest price.
    \item \textbf{Example:} Answering questions about today's news or weather forecasts.
\end{itemize}

\subsection*{Interacting with Databases}
LLMs can use tools to query and retrieve data from various databases.
\begin{itemize}
    \item \textbf{Example:} A customer service LLM might use a \texttt{get\_customer\_order\_status} tool that queries a CRM or order management system database 
    based on a customer's ID or order number.
\end{itemize}

\subsection*{Performing Actions}
Tools can enable LLMs to trigger actions in other systems.
\begin{itemize}
    \item \textbf{Example:} A personal assistant LLM could use a \texttt{send\_email} tool to draft and send an email based on user instructions, or a 
    \texttt{schedule\_meeting} tool to interact with a calendar API.
    \item \textbf{Example:} An LLM could use a tool to pay a bill through a financial service API.
\end{itemize}

\subsection*{Using Computational Tools}
LLMs are not inherently strong at precise mathematical calculations. Tools can offload these tasks to specialized engines.
\begin{itemize}
    \item \textbf{Example:} For a query like "What is 25\% of 789?", the LLM can call a calculator tool or even a more advanced computational engine like WolframAlpha.
\end{itemize}

\subsection*{Connecting with Enterprise Systems}
LLMs can integrate with internal business applications, such as CRMs, ERPs, or proprietary knowledge bases, via tools that wrap these systems' APIs. 
This allows LLMs to provide contextually relevant information based on specific company data.

By connecting LLMs to live data streams and actionable systems, tool use transforms them from sophisticated text generators into versatile components 
capable of participating actively in business processes and information workflows. 